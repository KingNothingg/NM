\section{Выводы}
В ходе выполнения курсового проекта я изучил метод сопряженных градиентов для решения СЛАУ.
Основной сложностью была корректная реализация класса разреженных матриц для хранения данных. Арифметические операции долго отлаживались, в связи с специфичным хранением данных (формат CSR).

Также, у меня не получился алгоритм для генерации случайной положительно определенной, разреженной и симметричной матрицы. Из-за этого, тесты я генерировал на языке python с помощью библиотеки sklearn.

В работе сравниваются методы сопряженных градиентов и Зейделя. Не уверен, что мой метод Зейделя реализован оптимально, но получилось так, что он работает на несколько порядков медленнее, чем метод сопряженных градиентов.

\pagebreak
