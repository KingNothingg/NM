\section{Исходный код}

Ниже приведена реализация класса разреженной матрицы:

\begin{lstlisting}[language=C++]
template<typename T>
class SparseMatrix {

private:
    std::vector<int> rows;
    std::vector<int> columns;
    std::vector<T> data;
    int m, n;

    void insert(int ind, int row, int col, T val) {
        if (data.empty()) {
            data = std::vector<T> (1, val);
            columns = std::vector<int> (1, col);
        } else {
            data.insert(data.begin() + ind, val);
            columns.insert(columns.begin() + ind, col);
        }
        for (int i = row + 1; i <= m; ++i) {
            rows[i]++;
        }
    }

    void remove(int index, int row) {
        data.erase(data.begin() + index);
        columns.erase(columns.begin() + index);

        for (int i = row + 1; i <= m; ++i) {
            rows[i]--;
        }
    }


public:
    SparseMatrix(int row, int col) {
        m = row;
        n = col;
        rows = std::vector<int> (row + 1, 0);
    }

    SparseMatrix(int _n) {
        m = _n;
        n = _n;
        rows = std::vector<int> (_n + 1, 0);
    }

    int size() const {
        if (n == m) {
            return n;
        } else {
            return 0;
        }
    }

    SparseMatrix(const SparseMatrix<T> &matrix) {
        n = matrix.n;
        m = matrix.m;
        rows = matrix.rows;
        columns = matrix.columns;
        data = matrix.data;
    }

    SparseMatrix(const std::vector<std::vector<T>> &matrix) {
        int count = 0;
        rows.clear();
        for (int i = 0; i < matrix.size(); ++i) {
            rows.push_back(count);
            for (int j = 0; j < matrix[i].size(); ++j) {
                if (matrix[i][j] != 0)
                    ++count;
            }
        }
        rows.push_back(count);
        columns.resize(count);
        data.resize(count);
        n = matrix[0].size();
        m = matrix.size();
        int k = 0;
        for (int i = 0; i < matrix.size(); ++i) {
            for (int j = 0; j < matrix[i].size(); ++j) {
                if (matrix[i][j] != 0) {
                 columns[k] = j;
                 data[k] = matrix[i][j];
                 ++k;
                }
            }
        }
    }

    T get(int row, int column) const {
        if (column > n || row > m) {
            throw "Trying to get non-existent element";
        }
        int currcol;
        for (int pos = rows[row]; pos < rows[row+1]; ++pos) {
            currcol = columns[pos];
            if (currcol == column) {
                return data[pos];
            } else if (currcol > column) {
                break;
            }
        }
        return T();
    }

    void swapRows(int n, int m) {

    }

    void set(T val, int row, int column) {
        if (column > n || row > m) {
            throw "Trying to set non-existent element";
        }
        int pos = rows[row];
        int currCol = 0;

        for (; pos < rows[row + 1]; ++pos) {
            currCol = columns[pos];
            if (currCol >= column) {
                break;
            }
        }

        if (currCol != column || data.empty() || pos >= data.size()) {
            insert(pos, row, column, val);
        } else if (val == T()) {
            remove(pos, row);
        } else {
            data[pos] = val;
        }
    }

    std::vector<T> operator *(const std::vector<T> &x) {
        if (n != x.size()) {
            throw "Invalid dimensions on multiply";
        }
        std::vector<T> result(m, T());
        if (!data.empty()) {
            for (int i = 0; i < m; ++i) {
                T sum = T();
                for (int j = rows[i]; j < rows[i + 1]; ++j) {
                    sum = sum + data[j] * x[columns[j]];
                }
                result[i] = sum;
            }
        }
        return result;
    }

    friend SparseMatrix<T> operator *(const SparseMatrix<T> &lhs, SparseMatrix<T> &mat) {
        if (lhs.n != mat.m) {
            throw "Invalid dimensions on multiply";
        }
        SparseMatrix<T> result(lhs.m, mat.n);
        T a;
        for (int i = 0; i < lhs.m; ++i) {
            for (int j = 0; j < mat.n; ++j) {
                a = T();
                for (int k = 0; k < lhs.n; ++k) {
                    a = a + lhs.get(i, k) * mat.get(k, j);
                }
                result.set(a, i, j);
            }
        }
        return result;
    }

    friend SparseMatrix<T> operator +(const SparseMatrix<T> &lhs, SparseMatrix<T> &rhs) {
        if (lhs.m != rhs.m || lhs.n != rhs.m) {
            throw "Invalid dimensions on addition";
        }
        SparseMatrix<T> result(lhs.m, lhs.n);
        for (int i = 0; i < lhs.m; ++i) {
            for (int j = 0; j < lhs.n; ++j) {
                result.set(lhs.get(i,j) + rhs.get(i,j), i, j);
            }
        }
        return result;
    }

    friend SparseMatrix<T> operator -(const SparseMatrix<T> &lhs, SparseMatrix<T> &rhs) {
        if (lhs.m != rhs.m || lhs.n != rhs.m) {
            throw "Invalid dimensions on substraction";
        }
        SparseMatrix<T> result(lhs.m, lhs.n);
        for (int i = 0; i < lhs.m; ++i) {
            for (int j = 0; j < lhs.n; ++j) {
                result.set(lhs.get(i,j) - rhs.get(i,j), i, j);
            }
        }
        return result;
    }

    friend std::ostream & operator << (std::ostream &os, const SparseMatrix<T> &matrix) {
        for (int i = 0; i < matrix.m; ++i) {
            for (int j = 0; j < matrix.n; ++j) {
                if (j != 0) {
                    os << " ";
                }
                os << matrix.get(i, j);
            }
            if (i < matrix.m - 1) {
                os << std::endl;
            }
        }
        return os;
    }
};
\end{lstlisting}

Ниже приведён листинг метода сопряженных градиентов:

\begin{lstlisting}[language=C++]
template <typename T>
class ConjGrad {
public:
    SparseMatrix<T> A;
    std::vector<T> b;
    double eps;

    ConjGrad(SparseMatrix<T> &_A, std::vector<T> &_b, double _eps) : A(_A), b(_b), eps(_eps) {};

    std::vector<T> solve(bool flag) {
        if (flag) {
            std::cout << "Матрица системы: \n" << A;
            std::cout << "Столбец свободных членов: \n" << b << '\n';
        }
        std::vector<T> x1(b.size(), 0);
        std::vector<T> r1 = b - (A * x1);
        std::vector<T> p1 = r1;
        int count = 0;
        do {
            std::vector<T> temp = A * p1;
            T alpha = (r1 * r1)/(temp * p1);
            std::vector<T> xk = x1 + (p1 * alpha);
            std::vector<T> rk = r1 - (temp) * alpha;
            T t = rk * rk;
            T t2 = r1 * r1;
            T beta = (rk * rk)/(r1 * r1);
            std::vector<T> pk = rk + (p1 * beta);
            x1  = xk;
            r1 = rk;
            p1 = pk;
            count++;
        } while (norm(r1) > eps);
        std::cout << "Количество итераций метода сопряженных градиентов: " << count << "\n";
        return x1;
    }
};
\end{lstlisting}
\pagebreak
