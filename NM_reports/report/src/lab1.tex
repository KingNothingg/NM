\section{Вычислительные методы линейной алгебры}

\subsection{LU-разложение матриц. Метод Гаусса}

\subsubsection{Постановка задачи}
Реализовать алгоритм LU-разложения матриц (с выбором главного элемента) в виде программы. Используя разработанное программное обеспечение, решить систему линейных алгебраических уравнений (СЛАУ). Для матрицы СЛАУ вычислить определитель и обратную матрицу.

\subsubsection{Консоль}
\begin{alltt}
4
-8 5 8 -6
2 7 -8 -1
-5 -4  1 -6
5 -9 -2 8
-144 25 -21 103
Решение системы: 
x1=9.000000
x2=-6.000000
x3=-6.000000
x4=-1.000000
Определитель: 1867.000000
Обратная матрица: 
-0.392608 -0.303160 -0.101768 -0.408677 
0.049813 0.043921 -0.077129 -0.014997 
-0.089448 -0.186395 -0.087306 -0.155865 
0.279057 0.192287 -0.044992 0.324585
\end{alltt}
\pagebreak

\subsubsection{Исходный код}
\lstinputlisting{../../NM1/lu.cpp}
\pagebreak

\subsection{Метод прогонки}

\subsubsection{Постановка задачи}
Реализовать метод прогонки в виде программы, задавая в качестве входных данных ненулевые элементы матрицы системы и вектор правых частей. Используя разработанное программное обеспечение, решить СЛАУ с трехдиагональной матрицей.

\subsubsection{Консоль}
\begin{alltt}
5
10 -1
-8 16 1
6 -16 6
-8 16 -5
5 -13
16 -110 24 -3 87
Решение системы: 
x1 = 1.000000
x2 = -6.000000
x3 = -6.000000
x4 = -6.000000
x5 = -9.000000
\end{alltt}
\pagebreak

\subsubsection{Исходный код}
\lstinputlisting{../../NM1_2/tridiag.cpp}
\pagebreak

\subsection{Итерационные методы решения СЛАУ}

\subsubsection{Постановка задачи}
Реализовать метод простых итераций и метод Зейделя в виде программ, задавая в качестве входных данных матрицу системы, вектор правых частей и точность вычислений. Используя разработанное программное обеспечение, решить СЛАУ. Проанализировать количество итераций, необходимое для достижения заданной точности.

\subsubsection{Консоль}
\begin{alltt}
4 0.000001
15 0 7 5
-3 -14 -6 1
-2 9 13 2
4 -1 3 9
176 -111 74 76
Метод простых итераций, решение системы:
x1 = 9.000000
x2 = 5.000000
x3 = 3.000000
x4 = 4.000000
Количество итераций: 88
Метод Зейделя, решение системы:
x1 = 9.000000
x2 = 5.000000
x3 = 3.000000
x4 = 4.000000
Количество итераций: 31

\end{alltt}
\pagebreak

\subsubsection{Исходный код}
\lstinputlisting{../../NM1_3/iter.cpp}
\pagebreak

\subsection{Метод вращений}

\subsubsection{Постановка задачи}
Реализовать метод вращений в виде программы, задавая в качестве входных данных матрицу и точность вычислений. Используя разработанное программное обеспечение, найти собственные значения и собственные векторы симметрических матриц. Проанализировать зависимость погрешности вычислений от числа итераций.

\subsubsection{Консоль}
\begin{alltt}
3 0.000001
-8 9 6
9 9 1
6 1 8
Собственные значения:
l1 = -13.141391
l2 = 14.757377
l3 = 7.384014
Собственные векторы: 
0.903164 0.429261 0.005498 
-0.356301 0.756677 -0.548168 
-0.239468 0.493127 0.836350 
Количество итераций: 7
\end{alltt}
\pagebreak

\subsubsection{Исходный код}
\lstinputlisting{../../NM1_4/rotation.cpp}
\pagebreak

\subsection{QR алгоритм}

\subsubsection{Постановка задачи}
Реализовать алгоритм QR – разложения матриц в виде программы. На его основе разработать программу, реализующую QR – алгоритм решения полной проблемы собственных значений произвольных матриц, задавая в качестве входных данных матрицу и точность вычислений. С использованием разработанного программного обеспечения найти собственные значения матрицы.

\subsubsection{Консоль}
\begin{alltt}
3 0.00001
0 -1 3
-1 6 -3
-8 4 2
Количество итераций: 27
Собственные значения:
l_1 = 2.31146 + i * 5.10345
l_2 = 2.31146 + i * -5.10345
l_3 = 3.37709
\end{alltt}
\pagebreak

\subsubsection{Исходный код}
\lstinputlisting{../../NM1_5/QR.cpp}
\pagebreak
