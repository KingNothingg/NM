\section{Численные методы решения нелинейных уравнений}

\subsection{Решение нелинейных уравнений}

\subsubsection{Постановка задачи}
Реализовать методы простой итерации и Ньютона решения нелинейных уравнений в виде программ, задавая в качестве входных данных точность вычислений. С использованием разработанного программного обеспечения найти положительный корень нелинейного уравнения (начальное приближение определить графически). Проанализировать зависимость погрешности вычислений от количества итераций.

\subsubsection{Консоль}
\begin{alltt}
-0.5 0 0.0000001

Метод простой итерации. Количество итераций: 10
x_0 = -0.474626594
Метод Ньютона. Количество итераций: 3
x_0 = -0.474626618
\end{alltt}
\pagebreak

\subsubsection{Исходный код}
\lstinputlisting{../../NM2_1/eqSolver.cpp}
\pagebreak

\subsection{Решение нелинейных систем уравнений}

\subsubsection{Постановка задачи}
Реализовать методы простой итерации и Ньютона решения систем нелинейных уравнений в виде программного кода, задавая в качестве входных данных точность вычислений. С использованием разработанного программного обеспечения решить систему нелинейных уравнений (при наличии нескольких решений найти то из них, в котором значения неизвестных являются положительными); начальное приближение определить графически. Проанализировать зависимость погрешности вычислений от количества итераций.

\subsubsection{Консоль}
\begin{alltt}
1 2
1.3 2.1
0.00001
Решение методом простых итераций: 1.275780 2.059608
Количество итераций: 8
Решение методом Ньютона: 1.275762 2.059607
Количество итераций: 5

\end{alltt}
\pagebreak

\subsubsection{Исходный код}
\lstinputlisting{../../NM2_2/solver.h}
\pagebreak
